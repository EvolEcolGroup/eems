\documentclass[draft,a4paper,10pt,DIV=15,titlepage,mpinclude=true]{scrartcl}
\usepackage[xetex]{graphicx}
\usepackage{amsmath,amsthm}
\usepackage[usenames,dvipsnames,svgnames,table]{xcolor}

\linespread{1.5}
\textwidth=480pt

\usepackage{hyperref}
\usepackage[T1]{fontenc}

% Comment out the next three lines to revert to Times New Roman
\usepackage{pxfonts}
\usepackage[scaled=0.86]{berasans}
\usepackage[scaled=0.96]{inconsolata}

\usepackage{afterpage}
\usepackage{verbatimbox}
\usepackage[final]{listings}


\lstdefinestyle{Cppcode}{
  language=c++,
  basicstyle=\footnotesize\ttfamily,
  numbers=left,
  stepnumber=1,
  numbersep=7pt,
  numberstyle=\tiny\color{gray},
  backgroundcolor=\color{black!3},
  showstringspaces=false,
  columns=flexible,
  rulecolor=\color{black!3},
  breaklines=true,
  breakatwhitespace=true,
  alsoletter={.,0,1,2,3,4,5,6,7,8,9},
  keywordstyle=\color{RoyalBlue},
  commentstyle=\color{YellowGreen},
  stringstyle=\color{ForestGreen},
  escapeinside={\#\%*}{*)},
  mathescape=true,
  alsoletter={.},
  morecomment=[l]{//},
  morecomment=[s]{/*}{*/},
  morestring=[b]',
  morekeywords={NULL,TRUE,FALSE,*,...},
}
\lstdefinestyle{Rcode}{
  language=R,
  basicstyle=\footnotesize\ttfamily,
  numbers=left,
  stepnumber=1,
  numbersep=7pt,
  numberstyle=\tiny\color{gray},
  backgroundcolor=\color{black!3},
  showstringspaces=false,
  columns=flexible,
  rulecolor=\color{black!3},
  breaklines=true,
  breakatwhitespace=true,
  alsoletter={.,0,1,2,3,4,5,6,7,8,9},
  keywordstyle=\color{RoyalBlue},
  commentstyle=\color{YellowGreen},
  stringstyle=\color{ForestGreen},
  escapeinside={\#\%*}{*)},
  mathescape=true,
  alsoletter={.},
  morecomment=[l]{//},
  morecomment=[s]{/*}{*/},
  morestring=[b]',
  morekeywords={NULL,TRUE,FALSE,*,...},
}
\lstset{literate=
    {0}{{{\color{BurntOrange}{0}}}}1
    {1}{{{\color{BurntOrange}{1}}}}1
    {2}{{{\color{BurntOrange}{2}}}}1
    {3}{{{\color{BurntOrange}{3}}}}1
    {4}{{{\color{BurntOrange}{4}}}}1
    {5}{{{\color{BurntOrange}{5}}}}1
    {6}{{{\color{BurntOrange}{6}}}}1
    {7}{{{\color{BurntOrange}{7}}}}1
    {8}{{{\color{BurntOrange}{8}}}}1
    {9}{{{\color{BurntOrange}{9}}}}1
    {.0}{{{\color{BurntOrange}{.0}}}}2
    {.1}{{{\color{BurntOrange}{.1}}}}2
    {.2}{{{\color{BurntOrange}{.2}}}}2
    {.3}{{{\color{BurntOrange}{.3}}}}2
    {.4}{{{\color{BurntOrange}{.4}}}}2
    {.5}{{{\color{BurntOrange}{.5}}}}2
    {.6}{{{\color{BurntOrange}{.6}}}}2
    {.7}{{{\color{BurntOrange}{.7}}}}2
    {.8}{{{\color{BurntOrange}{.8}}}}2
    {.9}{{{\color{BurntOrange}{.9}}}}2
    {\ }{{ }}{1}% handle the space
}


% Change the style of the captions and the headers
\usepackage[normal,bf,labelsep=quad]{caption}
\usepackage{subcaption}
\setkomafont{caption}{\sffamily}
\setkomafont{disposition}{\bfseries}


% Macro for finding figures
\graphicspath{{./Figures/}}

\usepackage{cleveref}
\captionsetup[subfigure]{subrefformat=simple,labelformat=simple}
\renewcommand\thesubfigure{\textbf{(\alph{subfigure})}}


%\newcommand{\keystring}[1]{{\color{red}\tt #1}}
\newcommand{\keystring}[1]{{\tt #1}}


\newcommand{\given}{\,|\,}
\newcommand{\specialrm}[1]{\mathrm{#1}}
\providecommand{\E}{\specialrm{E}}


\begin{document}

This short document explains the important difference between \keystring{bed2diffs-v1} and \keystring{bed2diffs-v2}.

\section{bed2diffs-v1}

Here the average difference between two samples, $i$ and $j$, is given by
\begin{align}
D_{ij} = \frac{1}{\big|M_{ij}\big|}\sum_{m\in M_{ij}}\big(z_{im} - z_{jm}\big)^2,
\end{align}
where $M_{ij}$ is the set of SNPs where both $i$ and $j$ are called, and $z_{im}$, $z_{jm}$ are the genotypes of individuals $i,j$ at marker $m$.

In practice, this computation is not guaranteed to produce an Euclidean distance matrix -- a nonnegative matrix with 0s on the diagonal, with exactly one positive eigenvalue \cite{Gower:1982fk} -- especially, if genotypes are not missing at random.

\section{bed2diffs-v2}
Alternatively,
\begin{align}
D_{ij} = \frac{1}{\big|M_{tot}\big|}\sum_{m\in M_{tot}}\big(z^*_{im} - z^*_{jm}\big)^2,
\end{align}
where $M_{tot}$ is the set of all markers and
\begin{align}
z^*_{im} = \begin{cases}
z_{im} & \text{if $z_{im}$ is called,}\\
\bar{z}_m & \text{otherwise,} \\
\end{cases}
\end{align}
where $\bar{z}_m$ is the average genotype at marker $m$.

This would produce an Euclidean distance matrix. Setting $z_{im}$ to the observed average at the marker $m$ is similar to the ``imputation'' often used before principal component analysis \cite{Price:2006kx}.

\bibliographystyle{apalike}
\bibliography{EEMS-biblio}


\end{document}
